
\documentclass[12pt]{article}
\usepackage[T2A]{fontenc}
\usepackage[english, russian]{babel} 

\usepackage{mathtools}
\usepackage[thinc]{esdiff}

\date{April 5, 2023}

\title{23. Резонанс}

% 23_resonance

\begin{document}

\maketitle

\section{Комплексные числа}

Формула Эйлера
\[
    e^{ix}=\cos x+i\sin x
\]

Функцию $F=F_0 \cos(\omega t-\Delta)$ будет рассматривать как действительную часть комплексного числа $F_0 e^{-i \Delta} e^{i \omega t}$. В физике не бывает комплексных сил, однако мы будет пользоваться данной записью для удобства
\[
    F=F_0 e^{-i \Delta} e^{i \omega t} = \hat{F} e^{i \omega t}
\]

Шляпка над буквой будет указывать что мы имеем дело с комплексным числом, в таком виде можно сразу описать амплитуду и сдвиг по фазе колебаний
\[
    \hat{F}=F_0 e^{-i \Delta}
\]

Будем решать уравнение, где на наш осциллятор действует внешняя сила $F$
\[
    m\diff[2]{x}{t}=-kx + F
\]

Будем предпологать что внешняя сила также осциллирует
\[
    \diff[2]{x}{t}+\frac{kx}{m}=\frac{F}{m}=\frac{F_0}{m} \cos\omega t
\]

\newpage

Перепишем уравнение сделав подстановку с комплексными числами, такую подстановку можно сделать не всегда, а только для \textit{линейных} уравнений содержащих $x$ в первой или нулевой степени. В таком случае можно выделить в исходном уравнении действительную и мнимую часть, при этом действительная часть будет в точности совпадать с исходным уравнением.
\[
    \diff[2]{x}{t}+\frac{kx}{m}=\frac{\hat{F} e^{i \omega t}}{m}
\]
в этом уравнении $x$ также комплексное число $x=\hat{x} e^{i \omega t}$, а каждое дифференцирование по времени равно умножению на $(i \omega)$. Мы применяем тут форму решения $x$ в виде $x=x_0\cos(\omega t+\Delta)$ или в комплексной форме $x=e^{i \Delta} e^{i \omega t} = \hat{x} e^{i \omega t}$ - грузик начинает колебаться с частотой действующей силы.

После дифференцирования и сокращения $e^{i \omega t}$ получаем
\[
    (i \omega)^2 \hat{x} + \frac{k\hat{x}}{m} = \frac{\hat{F}}{m}
\]

Откуда легко получить
\[
    \hat{x} = \frac{\hat{F}}{m(\omega_0^2-\omega^2)}
\]

Какого вида числа $\hat{F}$ и $\hat{x}$? $\hat{F}=F_0 e^{i \Delta_1}$ и $\hat{x}=x_0 e^{i \Delta_2}$ или это просто комплексные числа так как нет смысла говорить о фазе в данном случае?

\medskip

Этот же результат мы получали и раньше в главе 21.
\[
    x_0 = \frac{F_0}{m(\omega_0^2-\omega^2)}
\]


Грузик колеблется с частотой действующей силы, а амплитуда колебаний зависит от соотношения $\omega$ и $\omega_0$. Когда $\omega$ очень мала, грузик движется вслед за силой, если слишком быстро менять направление толчков, то грузик начинает двигаться в противоположном по отношению к силе направлению. При очень высокой частоте внешней силы грузик практически не двигается.

Так как $m(\omega_0^2-\omega^2)$ действительное число, \textit{фазовые углы} $F$ и $x$ совпадают(или отличаются на 180 градусов если $\omega^2 > \omega_0^2$).

Что имеется в виде под термином \textit{фазовый угол}, величина $\omega t+\Delta$ или просто $\Delta$ ?

Имеет ли тут вообще смысл говорить о сдвиге фазы $\Delta$ так как данные уравнения описывают уже устроявшийся процесс и нам не важно какая фаза была в начале? В начале не важно, но $\Delta$ может описывать устоявшуюся разницу в фазах между двумя уравнениями колебаний.

\section{Вынужденные колебания с торможением}





\end{document}

