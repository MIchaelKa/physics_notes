
\documentclass[12pt]{article}
\usepackage[T2A]{fontenc}
\usepackage[english, russian]{babel} 

\usepackage{mathtools}
\usepackage[thinc]{esdiff}

\usepackage[a4paper,margin=1in,footskip=0.5in]{geometry}
\usepackage{indentfirst}

\usepackage[bookmarks=true, colorlinks=false]{hyperref} 
\usepackage{bookmark}

\date{April 8, 2023}

\title{Электромагнитное Излучение}
% Electromagnetic Radiation
% electromagnetic_radiation

\begin{document}

\maketitle

\tableofcontents

\newpage

\section{Радиационное затухание}

\subsection{Радиационное затухание}

Модель электрона в атоме: заряд, закрепленный на пружине с собственной частотой \(\omega_0\).
Определим для такого осциллятора величину \(Q\), потеря энергии в данном случае вызвана так называемым радиационным сопротивлением или радиационным затуханием.

Для любой колеблющейся системы \textit{добротность} \(Q\) определяется как отношение энергии системы, к потере потери энергии за 1 радиан:
\[
    Q = \frac{W}{dW/d\phi}
\]
Запишем \(Q\) по-другому воспользуясь равенством
\[
    dW/d\phi = (dW/dt)/(d\phi/dt) = -(dW/dt)/(\omega)
\] 
\textit{Вопрос}: откуда появляется знак минус?

\medskip

Итого получаем:
\[
    Q = - \frac{\omega W}{dW/dt} \quad (32.8)
\]

Если \(Q\) задано то легко получить закон спадения энергии колебаний:
\[
    (dW/dt) = - \frac{\omega}{Q} W
\]
откуда следует что:
\[
    W = W_0 e^{-\omega t/Q}
\]

\medskip

Формула полного потока энергии, излучаемого зарядом. Мощность излучения.
\[
    P = \frac{q^2 \omega^4 x_0^2 }{12 \pi \epsilon_0 c^3} \quad (32.6)
\] 
Подставим (32.6) в (32.8) вместо \(dW/dt\), а в качестве полной энергии осциллятора возьмем \(W=\frac{1}{2}m_e\omega_0^2x_0^2\), где \(\omega_0\) это собственная частота излучения атома, а \(m_e\) - масса электрона.

После сокращений получаем формулу:
\[
    \frac{1}{Q} =
    \frac
    {4 \pi e^2}
    {2 \lambda m_e c^2}
    \quad (32.10)
\]
В формуле (32.10) \(e^2=\frac{q_e^2}{4\pi\epsilon_0}\), также мы воспользовались тем что \(2\pi\lambda=\omega_0/c\)

\end{document}