
\documentclass[12pt]{article}
\usepackage[T2A]{fontenc}
\usepackage[english, russian]{babel} 

\usepackage{mathtools}
\usepackage[thinc]{esdiff}

\usepackage[a4paper,margin=1in,footskip=0.5in]{geometry}
\usepackage{indentfirst}

\usepackage[bookmarks=true, colorlinks=false]{hyperref} 
\usepackage{bookmark}

\date{April 3-7, 2023}

\title{Теория колабаний}
% Oscillation theory
% oscillation_theory

\begin{document}

\maketitle

\tableofcontents

\section{Гармонический осциллятор}

\subsection{Гармонический осциллятор}

\subsubsection{Груз на пружине}

Гармонический осциллятор описывается линейным дифференциальным уравнением (ЛОДУ).

Простейший пример такой системы, груз массы $m$ на пружинке.
\[
m\diff[2]{x}{t}=-kx \quad (21.2)
\]

Функция $x=\cos(t)$ является решением данного уравнения если положить $k/m$ = 1.
Но каким образом мы можем все же учесть коэфициенты $k$ и $m$?

Попробуем $x=A\cos(t)$ и откроем важное свойство ЛОДУ: решение умноженное на \textit{константу} также является решением. Но мы по-прежнему не можем выразить коэфициенты $k$ и $m$.

Попробуем $x=\cos(\omega_0 t)$ и найдем $\omega_0^2=k/m$. Велечину $\omega_0 t$ часто называют \textit{фазой} движения.

Период полного колебания $t_0$, время за которое фаза изменяеися на $2\pi$ или $\omega_0t_0=2\pi$
\[
    t_0=2\pi\sqrt{m/k} \quad (21.5)
\]

Уравнение (21.2) определяет \textit{период} колебаний, но ничего не говорит нам о том \textit{как} началось движение, насколько мы оттянули пружинку, а также об амплитуде колабаний. Для этого нужно задать \textit{начальные условия}.

\subsubsection{Общее решение}

Нужно найти более общее решение уравнения (21.2). Общее решение должно допускать изменение начала отсчета времени, например
\[
    x=a\cos(\omega_0(t-t_1)) \quad (21.6a)
\]
или
\[
    x=a\cos(\omega_0t+\Delta) \quad (21.6b)
\]

Можно разложить $\cos(\omega_0t+\Delta)=\cos\omega_0t\cos\Delta-\sin\omega_0t\sin\Delta$ и записать:
\[
    x=A\cos\omega_0t+B\sin\omega_0t \quad (21.6c)
\]
где $A=a\cos\Delta$, a $B=-a\sin\Delta$

\medskip

Рассмотрим некоторые величины в уравнениях (21.6)
\\
$\omega_0$ - угловая частота, число радианов на которое фаза изменится за одну секунду, определяется дифференциальным уравнением (21.2)
\\
Другие величины не определяются дифференциальным уравнением, а зависят от начальных условий
\\
$a$ - амплитуда колебаний
\\
$\Delta$ - сдвиг фазы, иногда называют фазой
\\
Все вместе $\omega_0t+\Delta$ - удобно называть фазой

\subsection{Начальные условия}

$A$ и $B$ или $a$ и $\Delta$, показывают как началось движение. Эти значения можно определить из начальных условий, например, пусть в начальный момент времени $t=0$ грузик смещен от положения равновесия на велечину $x_0$ и имеет скорость  $v_0$.

Для получения коэфициентов $A$ и $B$ (а затем $a$ и $\Delta$) удобно пользоваться формулой (21.6c)

Для задания начальных условия для ЛОДУ второго порядка, мы должны задать значение самой функции во время $t=0$ и значение ее первой производной (скорости).

Мы не можем задать начальное значение второй производной или \textit{ускорения}, так как оно зависит от свойств пружины. Почему? Как это объясняется математически?

Посмотрим на (21.2)
\[
\diff[2]{x}{t}=-\frac{k}{m}x
\]
Из решения (21.2) получаются формулы
\[
    x=a\cos(\omega_0t+\Delta)
\]
\[
    v=-\omega_0a\sin(\omega_0t+\Delta)
\]
\[
    acc=-\omega_0^2a\cos(\omega_0t+\Delta)=-\omega_0^2x
\]

\subsection{Энергия осциллятора}

Если нет трения то в такой системе должна сохраняться энергия. Проверим это, для этого удобно воспользовать формулами:
\[
    x=a\cos(\omega_0t+\Delta)
\]
\[
    v=-\omega_0a\sin(\omega_0t+\Delta)
\]

Найдем потенциальную энергию $U$
\[
    U = 1/2kx^2 = 1/2k a^2 \cos^2(\omega_0t+\Delta)
\]

Найдем кинетическую энергию $T$
\[
    T = 1/2mv^2 = 1/2 m \omega_0^2 a^2 \sin^2(\omega_0t+\Delta)
\]

Изменение кинетической энергии противоположно изменению потенциальной энергии и наоброт, следовательно полная энерния должна быть постоянна. Запишем суммартную энергию помня о том что $\omega_0^2=k/m$, откуда $k=\omega_0^2 m$
\[
    T+U = 1/2 m \omega_0^2 a^2 (\cos^2(\omega_0t+\Delta) + \sin^2(\omega_0t+\Delta) = 1/2 m \omega_0^2 a^2
\]

\textit{Средняя} потенциальная энергия равна половине максимально и следовательно половине полной.

\section{Резонанс}

\subsection{Комплексные числа}

Формула Эйлера
\[
    e^{ix}=\cos x+i\sin x
\]

Функцию $F=F_0 \cos(\omega t-\Delta)$ будем рассматривать как действительную часть комплексного числа $F_0 e^{-i \Delta} e^{i \omega t}$. В физике не бывает комплексных сил, однако мы будет пользоваться данной записью для удобства
\[
    F=F_0 e^{-i \Delta} e^{i \omega t} = \hat{F} e^{i \omega t}
\]

Шляпка над буквой будет указывать что мы имеем дело с комплексным числом, в таком виде можно сразу описать амплитуду и сдвиг по фазе колебаний
\[
    \hat{F}=F_0 e^{-i \Delta}
\]

Будем решать уравнение, где на наш осциллятор действует внешняя сила $F$
\[
    m\diff[2]{x}{t}=-kx + F
\]

Будем предпологать что внешняя сила также осциллирует
\[
    \diff[2]{x}{t}+\frac{kx}{m}=\frac{F}{m}=\frac{F_0}{m} \cos\omega t \quad (23.2)
\]

% \newpage

Перепишем уравнение сделав подстановку с комплексными числами, такую подстановку можно сделать не всегда, а только для \textit{линейных} уравнений содержащих $x$ в первой или нулевой степени. В таком случае можно выделить в исходном уравнении действительную и мнимую часть, при этом действительная часть будет в точности совпадать с исходным уравнением.
\[
    \diff[2]{x}{t}+\frac{kx}{m}=\frac{\hat{F} e^{i \omega t}}{m}
\]
в этом уравнении $x$ также комплексное число $x=\hat{x} e^{i \omega t}$, а каждое дифференцирование по времени равно умножению на $(i \omega)$. Мы применяем тут форму решения $x$ в виде $x=x_0\cos(\omega t+\Delta)$ или в комплексной форме $x=e^{i \Delta} e^{i \omega t} = \hat{x} e^{i \omega t}$ - грузик начинает колебаться с частотой действующей силы.

После дифференцирования и сокращения $e^{i \omega t}$ получаем
\[
    (i \omega)^2 \hat{x} + \frac{k\hat{x}}{m} = \frac{\hat{F}}{m}
\]

откуда легко получить
\[
    \hat{x} = \frac{\hat{F}}{m(\omega_0^2-\omega^2)} \quad (23.5)
\]

\medskip

Грузик колеблется с частотой действующей силы, а амплитуда колебаний зависит от соотношения $\omega$ и $\omega_0$. Когда $\omega$ очень мала, грузик движется вслед за силой, если слишком быстро менять направление толчков, то грузик начинает двигаться в противоположном по отношению к силе направлению. При очень высокой частоте внешней силы грузик практически не двигается.

Так как $m(\omega_0^2-\omega^2)$ действительное число, \textit{фазовые углы} $F$ и $x$ совпадают(или отличаются на 180 градусов если $\omega^2 > \omega_0^2$).

\subsubsection{Вопросы}

\textit{Вопрос}: Какого вида числа $\hat{F}$ и $\hat{x}$? $\hat{F}=F_0 e^{i \Delta_1}$ и $\hat{x}=x_0 e^{i \Delta_2}$ или это просто комплексные числа так как нет смысла говорить о фазе в данном случае? \textit{Ответ}: Решая эту же задачу в Главе 21, без использования комплексных чисел, мы не учитывали сдвиг по фазе в решении \(x\) и получили:
\[
    x_0 = \frac{F_0}{m(\omega_0^2-\omega^2)}
\]
теперь мы имеем:
\[
    x_0 e^{i \Delta_1} = \frac{F_0}{m(\omega_0^2-\omega^2)} e^{i \Delta_2}
\]

\medskip

\textit{Вопрос}: Что имеется в виде под термином \textit{фазовый угол}, величина $\omega t+\Delta$ или просто $\Delta$? \textit{Ответ}: Под фазовыми углами имеется ввиду величины \(\Delta_1\) и \(\Delta_2\), если они отличаются на 180 градусов, то \(e^{i(\Delta+\pi)} = e^{i(\Delta)}e^{i(\pi)} = -e^{i(\Delta)}\)

\medskip

\textit{Вопрос}: Значит ли это что мы всегда получаем положительную амплитуду из уравнения (23.5)?

\medskip

\textit{Вопрос}: Имеет ли тут вообще смысл говорить о сдвиге фазы $\Delta$ так как данные уравнения описывают уже устроявшийся процесс и нам не важно какая фаза была в начале?
\textit{Ответ}: $\Delta$ описывает устоявшуюся разницу в фазах между двумя уравнениями колебаний.

\textit{Упражение}: Попробовать представить как такое возможно, что \textit{если слишком быстро менять направление толчков, то грузик начинает двигаться в противоположном по отношению к силе направлению}.

\subsection{Вынужденные колебания с торможением}

Изменим формулу (23.2) так чтобы учесть трение
\[
    m \diff[2]{x}{t} + c \diff{x}{t} + kx = F \quad (23.6)
\]
если положить $c=m\gamma$ и $k=m\omega_0^2$ и поделить обе части на $m$ то получим:
\[
    \diff[2]{x}{t} + \gamma \diff{x}{t} + \omega_0^2 x = \frac{F}{m} \quad (23.6a)
\]
применим комплексные числа:
\[
    e^{i\omega t}[(i\omega)^2 \hat{x} + \gamma(i\omega)\hat{x} + \omega_0^2 \hat{x}] = \frac{\hat
    F}{m} e^{i\omega t} \quad (23.7)
\]
откуда легко найдем \textit{отклик} осциллятора
\[
    \hat{x}=\frac{\hat{F}}{m(\omega_0^2-\omega^2+i\gamma\omega)} \quad (23.8)    
\]
формулу (23.8) иногда называют "резонансной"

\medskip

Обозначим множитель перед \(\hat{F}\) через \(R\)
\[
    R=\frac{1}{m(\omega_0^2-\omega^2+i\gamma\omega)}
\]
тогда
\[
    \hat{x}=\hat{F}R \quad (23.9) 
\]

\medbreak

Множитель \(R\) можно записать в виде \(p+iq\) или \(\rho e^{i \theta}\). Запишем в виде \(\rho e^{i \theta}\) и посмотрим к чему это приведет.
\[
    \hat{x}=R\hat{F}=\rho e^{i\theta} F_0e^{i \Delta}=\rho F_0e^{i (\theta+\Delta)}
\]
Помня что \(x=\hat{x}e^{i \omega t}\) найдем:
\[
    x=\rho F_0 \cos(\omega t + \Delta + \theta) \quad (23.10)
\]
из этой формулы видно что \(\rho\) и \(\theta\) - это величина и фазовый сдвиг отклика.

\medbreak

Найдем значение \(\rho\) и \(\theta\). Из свойств комплексных чисел мы значем что если \(r\) модуль комплексного числа \(a=x+iy=r e^{i \theta}\), то \(r^2=aa^*\), а \(\tg \theta = \frac{y}{x} \)

Вычислим \(\rho\)
\[
    \rho^2 = \frac{1}{m^2(\omega_0^2-\omega^2+i\gamma\omega)(\omega_0^2-\omega^2-i\gamma\omega)} =
\]
\[
    = \frac{1}{m^2[(\omega_0^2-\omega^2)^2+\gamma^2\omega^2]} \quad (23.11)
\]

Вычислим \(\theta\)
\[
    \frac{1}{R} = \frac{1}{\rho e^{i \theta}} = \frac{1}{\rho}e^{- i \theta} = m(\omega_0^2-\omega^2+i\gamma\omega)
\]
заметим что перед \(\theta\) появился знак минус, а \( \tg (-\theta)=-\tg \theta \), тогда
\[
    \tg \theta = -\frac{\gamma\omega}{\omega_0^2-\omega^2} \quad (23.12)
\]

Из (23.12) видно что угол  \(\theta\) отрицательный при любых значениях \(\omega\), то есть смещение \(x\) отстает по фазе от силы \(F\)

График зависимости \(\rho^2\) от частоты \(\omega\) называют \textit{резонансной} кривой. \(\rho^2\) в физике интереснее чем \(\rho\), так как \(\rho^2\) пропорционально квадрату амплитуды, а значит той \textit{энергии} которую передает осциллятору внешняя сила.

Иногда удобней работать с упрощенной версией формулы (23.8). При малых \(\gamma\) наиболее интересная область резонансной кривой находится около частоты \(\omega=\omega_0\).\\
Если \(\gamma<<\omega_0\) и \(\omega \approx \omega_0\)
\[
    \hat{x} \approx \frac{\hat{F}}{2m\omega_0(\omega_0-\omega+i\gamma/2)}
\]
\[
    \rho^2 \approx \frac{1}{4m\omega_0^2[(\omega_0-\omega)^2+\gamma^2/4]}
\]

Интресен следующий вопрос: при каком расстоянии от \(\omega_0\) на резонансной кривой расположены частоты которым соответствует \(\rho^2\) вдвое меньше максимального? Можно показать что при очень малом \(\gamma\) эти точки отстоят друг от друга на расстоянии \(\Delta \omega = \gamma\). Это значит что резонанс становится все более острым по мере того как влияние трения становится все слабее и слабее.

Другой мерой ширины резонанса можется служить \textit{добротность} \(Q = \omega_0 / \gamma\)


\end{document}

