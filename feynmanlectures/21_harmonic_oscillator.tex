
\documentclass[12pt]{article}
\usepackage[T2A]{fontenc}
\usepackage[english, russian]{babel} 

\usepackage{mathtools}
\usepackage[thinc]{esdiff}

\date{April 3-5, 2023}

\title{21. Гармонический осциллятор}

% 21_harmonic_oscillator

\begin{document}

\maketitle

\section{Гармонический осциллятор}

\subsection{Груз на пружине}

Гармонический осциллятор описывается линейным дифференциальным уравнением (ЛОДУ).

Простейший пример такой системы, груз массы $m$ на пружинке.
\[
m\diff[2]{x}{t}=-kx \quad (21.2)
\]

Функция $x=\cos(t)$ является решением данного уравнения если положить $k/m$ = 1.
Но каким образом мы можем все же учесть коэфициенты $k$ и $m$?

Попробуем $x=A\cos(t)$ и откроем важное свойство ЛОДУ: решение умноженное на \textit{константу} также является решением. Но мы по-прежнему не можем выразить коэфициенты $k$ и $m$.

Попробуем $x=\cos(\omega_0 t)$ и найдем $\omega_0^2=k/m$. Велечину $\omega_0 t$ часто называют \textit{фазой} движения.

Период полного колебания $t_0$, время за которое фаза изменяеися на $2\pi$ или $\omega_0t_0=2\pi$
\[
    t_0=2\pi\sqrt{m/k} \quad (21.5)
\]

Уравнение (21.2) определяет \textit{период} колебаний, но ничего не говорит нам о том \textit{как} началось движение, насколько мы оттянули пружинку, а также об амплитуде колабаний. Для этого нужно задать \textit{начальные условия}.

\subsection{Общее решение}

Нужно найти более общее решение уравнения (21.2). Общее решение должно допускать изменение начала отсчета времени, например
\[
    x=a\cos(\omega_0(t-t_1)) \quad (21.6a)
\]
или
\[
    x=a\cos(\omega_0t+\Delta) \quad (21.6b)
\]

Можно разложить $\cos(\omega_0t+\Delta)=\cos\omega_0t\cos\Delta-\sin\omega_0t\sin\Delta$ и записать:
\[
    x=A\cos\omega_0t+B\sin\omega_0t \quad (21.6c)
\]
где $A=a\cos\Delta$, a $B=-a\sin\Delta$

\medskip

Рассмотрим некоторые величины в уравнениях (21.6)
\\
$\omega_0$ - угловая частота, число радианов на которое фаза изменится за одну секунду, определяется дифференциальным уравнением (21.2)
\\
Другие величины не определяются дифференциальным уравнением, а зависят от начальных условий
\\
$a$ - амплитуда колебаний
\\
$\Delta$ - сдвиг фазы, иногда называют фазой
\\
Все вместе $\omega_0t+\Delta$ - удобно называть фазой

\section{Начальные условия}

$A$ и $B$ или $a$ и $\Delta$, показывают как началось движение. Эти значения можно определить из начальных условий, например, пусть в начальный момент времени $t=0$ грузик смещен от положения равновесия на велечину $x_0$ и имеет скорость  $v_0$.

Для получения коэфициентов $A$ и $B$ (а затем $a$ и $\Delta$) удобно пользоваться формулой (21.6c)

% \pagebreak
\newpage
Для задания начальных условия для ЛОДУ второго порядка, мы должны задать значение самой функции во время $t=0$ и значение ее первой производной (скорости).

Мы не можем задать начальное значение второй производной или \textit{ускорения}, так как оно зависит от свойств пружины. Почему? Как это объясняется математически?

Посмотрим на (21.2)
\[
\diff[2]{x}{t}=-\frac{k}{m}x
\]
Из решения (21.2) получаются формулы
\[
    x=a\cos(\omega_0t+\Delta)
\]
\[
    v=-\omega_0a\sin(\omega_0t+\Delta)
\]
\[
    acc=-\omega_0^2a\cos(\omega_0t+\Delta)=-\omega_0^2x
\]

\section{Энергия осциллятора}

Если нет трения то в такой системе должна сохраняться энергия. Проверим это, для этого удобно воспользовать формулами:
\[
    x=a\cos(\omega_0t+\Delta)
\]
\[
    v=-\omega_0a\sin(\omega_0t+\Delta)
\]

Найдем потенциальную энергию $U$
\[
    U = 1/2kx^2 = 1/2k a^2 \cos^2(\omega_0t+\Delta)
\]

Найдем кинетическую энергию $T$
\[
    T = 1/2mv^2 = 1/2 m \omega_0^2 a^2 \sin^2(\omega_0t+\Delta)
\]

Изменение кинетической энергии противоположно изменению потенциальной энергии и наоброт, следовательно полная энерния должна быть постоянна. Запишем суммартную энергию помня о том что $\omega_0^2=k/m$, откуда $k=\omega_0^2 m$
\[
    T+U = 1/2 m \omega_0^2 a^2 (\cos^2(\omega_0t+\Delta) + \sin^2(\omega_0t+\Delta) = 1/2 m \omega_0^2 a^2
\]

\textit{Средняя} потенциальная энергия равна половине максимально и следовательно половине полной.

\end{document}

