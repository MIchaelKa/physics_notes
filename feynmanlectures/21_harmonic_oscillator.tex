
\documentclass[12pt]{article}
\usepackage[T2A]{fontenc}
\usepackage[english, russian]{babel} 

\usepackage{mathtools}
\usepackage[thinc]{esdiff}

\date{April 3, 2023}

\title{21. Гармонический осциллятор}

% 21_harmonic_oscillator

\begin{document}

\maketitle

\section{Гармонический осциллятор}

Описывается линейным дифференциальным уравнением (ЛОДУ).

Простейший пример такой системы, груз массы $m$ на пружинке.
\[
m\diff[2]{x}{t}=-kx
\]

Функция $x=\cos(t)$ является решением данного уравнения если положить $k/m$ = 1.
Но каким образом мы можем все же учесть коэфициенты $k$ и $m$?

Попробуем $x=A\cos(t)$ и откроем важное свойство ЛОДУ: решение умноженное на \textbf{константу} также является решением. Но мы по-прежнему не можем выразить коэфициенты $k$ и $m$.

Попробуем $x=\cos(\omega_0 t)$ и найдем $\omega_0^2=k/m$. Велечину $\omega_0 t$ часто называют \textbf{фазой} движения.

Период полного колебания $t_0$, время за которое фаза изменяеися на $2\pi$ или $\omega_0t_0=2\pi$
\[
t_0=2\pi\sqrt{m/k}
\]
Это уравнение ничего не говорит нам о том \textbf{как} началось движение, насколько мы оттянули пружинку, а также об амплитуде колабаний. Для этого нужно задать \textbf{начальные условия}.




\end{document}
